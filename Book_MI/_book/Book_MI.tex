\documentclass[]{book}
\usepackage{lmodern}
\usepackage{amssymb,amsmath}
\usepackage{ifxetex,ifluatex}
\usepackage{fixltx2e} % provides \textsubscript
\ifnum 0\ifxetex 1\fi\ifluatex 1\fi=0 % if pdftex
  \usepackage[T1]{fontenc}
  \usepackage[utf8]{inputenc}
\else % if luatex or xelatex
  \ifxetex
    \usepackage{mathspec}
  \else
    \usepackage{fontspec}
  \fi
  \defaultfontfeatures{Ligatures=TeX,Scale=MatchLowercase}
\fi
% use upquote if available, for straight quotes in verbatim environments
\IfFileExists{upquote.sty}{\usepackage{upquote}}{}
% use microtype if available
\IfFileExists{microtype.sty}{%
\usepackage{microtype}
\UseMicrotypeSet[protrusion]{basicmath} % disable protrusion for tt fonts
}{}
\usepackage[margin=1in]{geometry}
\usepackage{hyperref}
\hypersetup{unicode=true,
            pdftitle={Applied Missing data analysis with SPSS and R(Studio)},
            pdfauthor={Martijn Heymans and Iris Eekhout},
            pdfborder={0 0 0},
            breaklinks=true}
\urlstyle{same}  % don't use monospace font for urls
\usepackage{natbib}
\bibliographystyle{apalike}
\usepackage{color}
\usepackage{fancyvrb}
\newcommand{\VerbBar}{|}
\newcommand{\VERB}{\Verb[commandchars=\\\{\}]}
\DefineVerbatimEnvironment{Highlighting}{Verbatim}{commandchars=\\\{\}}
% Add ',fontsize=\small' for more characters per line
\usepackage{framed}
\definecolor{shadecolor}{RGB}{248,248,248}
\newenvironment{Shaded}{\begin{snugshade}}{\end{snugshade}}
\newcommand{\KeywordTok}[1]{\textcolor[rgb]{0.13,0.29,0.53}{\textbf{#1}}}
\newcommand{\DataTypeTok}[1]{\textcolor[rgb]{0.13,0.29,0.53}{#1}}
\newcommand{\DecValTok}[1]{\textcolor[rgb]{0.00,0.00,0.81}{#1}}
\newcommand{\BaseNTok}[1]{\textcolor[rgb]{0.00,0.00,0.81}{#1}}
\newcommand{\FloatTok}[1]{\textcolor[rgb]{0.00,0.00,0.81}{#1}}
\newcommand{\ConstantTok}[1]{\textcolor[rgb]{0.00,0.00,0.00}{#1}}
\newcommand{\CharTok}[1]{\textcolor[rgb]{0.31,0.60,0.02}{#1}}
\newcommand{\SpecialCharTok}[1]{\textcolor[rgb]{0.00,0.00,0.00}{#1}}
\newcommand{\StringTok}[1]{\textcolor[rgb]{0.31,0.60,0.02}{#1}}
\newcommand{\VerbatimStringTok}[1]{\textcolor[rgb]{0.31,0.60,0.02}{#1}}
\newcommand{\SpecialStringTok}[1]{\textcolor[rgb]{0.31,0.60,0.02}{#1}}
\newcommand{\ImportTok}[1]{#1}
\newcommand{\CommentTok}[1]{\textcolor[rgb]{0.56,0.35,0.01}{\textit{#1}}}
\newcommand{\DocumentationTok}[1]{\textcolor[rgb]{0.56,0.35,0.01}{\textbf{\textit{#1}}}}
\newcommand{\AnnotationTok}[1]{\textcolor[rgb]{0.56,0.35,0.01}{\textbf{\textit{#1}}}}
\newcommand{\CommentVarTok}[1]{\textcolor[rgb]{0.56,0.35,0.01}{\textbf{\textit{#1}}}}
\newcommand{\OtherTok}[1]{\textcolor[rgb]{0.56,0.35,0.01}{#1}}
\newcommand{\FunctionTok}[1]{\textcolor[rgb]{0.00,0.00,0.00}{#1}}
\newcommand{\VariableTok}[1]{\textcolor[rgb]{0.00,0.00,0.00}{#1}}
\newcommand{\ControlFlowTok}[1]{\textcolor[rgb]{0.13,0.29,0.53}{\textbf{#1}}}
\newcommand{\OperatorTok}[1]{\textcolor[rgb]{0.81,0.36,0.00}{\textbf{#1}}}
\newcommand{\BuiltInTok}[1]{#1}
\newcommand{\ExtensionTok}[1]{#1}
\newcommand{\PreprocessorTok}[1]{\textcolor[rgb]{0.56,0.35,0.01}{\textit{#1}}}
\newcommand{\AttributeTok}[1]{\textcolor[rgb]{0.77,0.63,0.00}{#1}}
\newcommand{\RegionMarkerTok}[1]{#1}
\newcommand{\InformationTok}[1]{\textcolor[rgb]{0.56,0.35,0.01}{\textbf{\textit{#1}}}}
\newcommand{\WarningTok}[1]{\textcolor[rgb]{0.56,0.35,0.01}{\textbf{\textit{#1}}}}
\newcommand{\AlertTok}[1]{\textcolor[rgb]{0.94,0.16,0.16}{#1}}
\newcommand{\ErrorTok}[1]{\textcolor[rgb]{0.64,0.00,0.00}{\textbf{#1}}}
\newcommand{\NormalTok}[1]{#1}
\usepackage{longtable,booktabs}
\usepackage{graphicx,grffile}
\makeatletter
\def\maxwidth{\ifdim\Gin@nat@width>\linewidth\linewidth\else\Gin@nat@width\fi}
\def\maxheight{\ifdim\Gin@nat@height>\textheight\textheight\else\Gin@nat@height\fi}
\makeatother
% Scale images if necessary, so that they will not overflow the page
% margins by default, and it is still possible to overwrite the defaults
% using explicit options in \includegraphics[width, height, ...]{}
\setkeys{Gin}{width=\maxwidth,height=\maxheight,keepaspectratio}
\IfFileExists{parskip.sty}{%
\usepackage{parskip}
}{% else
\setlength{\parindent}{0pt}
\setlength{\parskip}{6pt plus 2pt minus 1pt}
}
\setlength{\emergencystretch}{3em}  % prevent overfull lines
\providecommand{\tightlist}{%
  \setlength{\itemsep}{0pt}\setlength{\parskip}{0pt}}
\setcounter{secnumdepth}{5}
% Redefines (sub)paragraphs to behave more like sections
\ifx\paragraph\undefined\else
\let\oldparagraph\paragraph
\renewcommand{\paragraph}[1]{\oldparagraph{#1}\mbox{}}
\fi
\ifx\subparagraph\undefined\else
\let\oldsubparagraph\subparagraph
\renewcommand{\subparagraph}[1]{\oldsubparagraph{#1}\mbox{}}
\fi

%%% Use protect on footnotes to avoid problems with footnotes in titles
\let\rmarkdownfootnote\footnote%
\def\footnote{\protect\rmarkdownfootnote}

%%% Change title format to be more compact
\usepackage{titling}

% Create subtitle command for use in maketitle
\newcommand{\subtitle}[1]{
  \posttitle{
    \begin{center}\large#1\end{center}
    }
}

\setlength{\droptitle}{-2em}

  \title{Applied Missing data analysis with SPSS and R(Studio)}
    \pretitle{\vspace{\droptitle}\centering\huge}
  \posttitle{\par}
    \author{Martijn Heymans and Iris Eekhout}
    \preauthor{\centering\large\emph}
  \postauthor{\par}
      \predate{\centering\large\emph}
  \postdate{\par}
    \date{2018-09-03}

\usepackage{booktabs}

\usepackage{amsthm}
\newtheorem{theorem}{Theorem}[chapter]
\newtheorem{lemma}{Lemma}[chapter]
\theoremstyle{definition}
\newtheorem{definition}{Definition}[chapter]
\newtheorem{corollary}{Corollary}[chapter]
\newtheorem{proposition}{Proposition}[chapter]
\theoremstyle{definition}
\newtheorem{example}{Example}[chapter]
\theoremstyle{definition}
\newtheorem{exercise}{Exercise}[chapter]
\theoremstyle{remark}
\newtheorem*{remark}{Remark}
\newtheorem*{solution}{Solution}
\begin{document}
\maketitle

{
\setcounter{tocdepth}{1}
\tableofcontents
}
\chapter*{Preface}\label{preface}
\addcontentsline{toc}{chapter}{Preface}

The attention for missing data is growing and so will be the application
of methods to solve the missing data problem. From our experience,
researchers with missing data still find it difficult to reserve time to
evaluate the missing data and from that to find a reasonable solution to
handle their missing data for their main data analysis. This manual is
developed for researchers that are looking for a solution of their
missing data problem or want to learn more about missing data. The
manual is developed as a result of a missing data course that we give.
Further, we are also active in providing statistical advice in general
and more specific about missing data. Because our time to give advice is
mostly limited we wanted to give researchers a practical guide to help
them get started with their missing data problem. Leading methodologists
and statisticians and leading journals have published papers about the
problems of missing data and warned researchers to take missing data
seriously (Sterne et al., BMJ 2009, Little et al. NEJM 2012, Peng et al.
2015, JAMA). Hopefully this manual will help researchers to find the
best solution for their missing data problem. We hope you will enjoy
this manual and that you learn from it, at least to take missing data
seriously and that you will use recommended methods to solve your
missing data problem.

\section{The goal of this Manual}\label{the-goal-of-this-manual}

In this manual the software packages SPSS and R play a central role. The
combination of these two software packages may seem a coincidence, but
it is not. For a long time, SPSS was the most popular software package
worldwide to do statistical data analysis. Currently, R is growing in
popularity fast and will probably become one of the most popular
Software packages to do data analysis. Also for applied researchers.
Both SPSS and R have their advantages and disadvantages. An advantage of
SPSS is that it is a user-friendly software package compared to R and
works with windows where you can for example drag your variables to.
Subsequently, you can click the OK button and the statistical analysis
procedure you prespecified gives you the output results. A disadvantage
of SPSS may be that you are overloaded with statistical output that may
not all needed to answer your research question. Compared to SPSS you
could say that R is a more user-unfriendly software package where you
need to use R code to activate statistical procedures and to get
statistical results. R output will show more specific results, without
extra information. Furthermore, R works much faster when it comes to
running statistical procedures by using 1 or 2 lines of R code, compared
to visiting a couple of windows in SPSS to activate the same statistical
test. There is one other advantage of R and that is, that it is open
source. This makes it possible for applied researchers to follow the
calculations of complex procedures as the estimation of missing values
closely along the line. You could say that R brings you to the heart of
the matter. With R it is possible to turn complex data analysis
functions and formula´s into computer code that can be used by everybody
and vice versa. Because it is open source, you are able to read the code
that is used for the analysis and to relate that code or pieces of code
to the statistical output. This makes it possible to evaluate step by
step the code and thus the statistical procedures and relate them to the
subsequent results. You can copy specific parts of code from functions
that others have written and evaluate what happens. This is one of the
major advantages of R if you compare it to the closed source statistical
package SPSS. R brings you a big learning environment when it comes to
the understanding of all kind of statistical procedures as missing data
analysis.

\section{Multiple Imputation in SPSS and
R}\label{multiple-imputation-in-spss-and-r}

Multiple Imputation (MI) is a procedure that is developed in the 1970's
by Donald Rubin. Later, around the 1990´s Multiple imputation was
further developed and became more popular. For a long time, MI was only
available for S-Plus and R software (S-plus is the commercial
alternative of R), where it was further developed by Stef van Buuren, a
statistician from TNO, Leiden, The Netherlands. For a long time, it was
not possible to do MI analysis in SPSS because it was not available in
SPSS. So, it was far out of reach for applied researchers for a long
time. It became available from SPSS version 17. From that time MI is now
used more by applied researchers. In this manual the handling of missing
data is the main topic. We will also show how to apply these methods in
both software packages SPSS and R. To apply the imputation methods that
are discussed both software packages make use of random starting
procedures. SPSS and R use for that intern random number generators.
Because these are different, result might slightly differ. Our intention
is not to compare the software packages SPSS and R and their output
resultys. Both are trustful packages, it is more the estimation
procedures that might lead to the differences. The imputation methods,
will be applied in SPSS version 24 and with R software version 3.4.3.
The R examples will be presented by using the output from RStudio
version (version 1.1.383 -- © 2009-2017 RStudio, Inc.). RStudio is an
integrated development environment (IDE) for R. RStudio includes a wide
range of productivity enhancing features and runs on all major
platforms. As already stated, R allows you to program the statistical
formula's yourself. We have therefore chosen to explain the formula's in
more detail in combination with the application in R. The more applied
researchers will be satisfied with the explanation and application of
methods in SPSS.

\section{Notation and annotation in this
manual}\label{notation-and-annotation-in-this-manual}

The name of R packages, libraries and functions can be recognized by
using Courier new lettertype, for example the package mice will be
written as mice.

R code of the procedures used in the manual is marked grey and the
explanation in these grey parts can be found in the grey parts itself
annotated by the \# symbol. The lines that start with the symbol
\textgreater{} are R Code lines that have been running in the R Console
in RStudio. Example:

\textbf{R code XX}

\begin{Shaded}
\begin{Highlighting}[]
\CommentTok{# Activate the foreign package and read in the SPSS dataset}

\KeywordTok{library}\NormalTok{(foreign)}
\NormalTok{dataset <-}\StringTok{ }\KeywordTok{read.spss}\NormalTok{(}\DataTypeTok{file=}\StringTok{"Backpain 50 missing.sav"}\NormalTok{, }\DataTypeTok{to.data.frame=}\NormalTok{T)}
\end{Highlighting}
\end{Shaded}

\begin{verbatim}
## re-encoding from UTF-8
\end{verbatim}

\chapter{Introduction}\label{intro}

You can label chapter and section titles using \texttt{\{\#label\}}
after them, e.g., we can reference Chapter \ref{intro}. If you do not
manually label them, there will be automatic labels anyway, e.g.,
Chapter \ref{methods}.

Figures and tables with captions will be placed in \texttt{figure} and
\texttt{table} environments, respectively.

\begin{Shaded}
\begin{Highlighting}[]
\KeywordTok{par}\NormalTok{(}\DataTypeTok{mar =} \KeywordTok{c}\NormalTok{(}\DecValTok{4}\NormalTok{, }\DecValTok{4}\NormalTok{, .}\DecValTok{1}\NormalTok{, .}\DecValTok{1}\NormalTok{))}
\KeywordTok{plot}\NormalTok{(pressure, }\DataTypeTok{type =} \StringTok{'b'}\NormalTok{, }\DataTypeTok{pch =} \DecValTok{19}\NormalTok{)}
\end{Highlighting}
\end{Shaded}

\begin{figure}

{\centering \includegraphics[width=0.8\linewidth]{Book_MI_files/figure-latex/nice-fig-1} 

}

\caption{Here is a nice figure!}\label{fig:nice-fig}
\end{figure}

Reference a figure by its code chunk label with the \texttt{fig:}
prefix, e.g., see Figure \ref{fig:nice-fig}. Similarly, you can
reference tables generated from \texttt{knitr::kable()}, e.g., see Table
\ref{tab:nice-tab}.

\begin{Shaded}
\begin{Highlighting}[]
\NormalTok{knitr}\OperatorTok{::}\KeywordTok{kable}\NormalTok{(}
  \KeywordTok{head}\NormalTok{(iris, }\DecValTok{20}\NormalTok{), }\DataTypeTok{caption =} \StringTok{'Here is a nice table!'}\NormalTok{,}
  \DataTypeTok{booktabs =} \OtherTok{TRUE}
\NormalTok{)}
\end{Highlighting}
\end{Shaded}

\begin{table}

\caption{\label{tab:nice-tab}Here is a nice table!}
\centering
\begin{tabular}[t]{rrrrl}
\toprule
Sepal.Length & Sepal.Width & Petal.Length & Petal.Width & Species\\
\midrule
5.1 & 3.5 & 1.4 & 0.2 & setosa\\
4.9 & 3.0 & 1.4 & 0.2 & setosa\\
4.7 & 3.2 & 1.3 & 0.2 & setosa\\
4.6 & 3.1 & 1.5 & 0.2 & setosa\\
5.0 & 3.6 & 1.4 & 0.2 & setosa\\
\addlinespace
5.4 & 3.9 & 1.7 & 0.4 & setosa\\
4.6 & 3.4 & 1.4 & 0.3 & setosa\\
5.0 & 3.4 & 1.5 & 0.2 & setosa\\
4.4 & 2.9 & 1.4 & 0.2 & setosa\\
4.9 & 3.1 & 1.5 & 0.1 & setosa\\
\addlinespace
5.4 & 3.7 & 1.5 & 0.2 & setosa\\
4.8 & 3.4 & 1.6 & 0.2 & setosa\\
4.8 & 3.0 & 1.4 & 0.1 & setosa\\
4.3 & 3.0 & 1.1 & 0.1 & setosa\\
5.8 & 4.0 & 1.2 & 0.2 & setosa\\
\addlinespace
5.7 & 4.4 & 1.5 & 0.4 & setosa\\
5.4 & 3.9 & 1.3 & 0.4 & setosa\\
5.1 & 3.5 & 1.4 & 0.3 & setosa\\
5.7 & 3.8 & 1.7 & 0.3 & setosa\\
5.1 & 3.8 & 1.5 & 0.3 & setosa\\
\bottomrule
\end{tabular}
\end{table}

You can write citations, too. For example, we are using the
\textbf{bookdown} package \citep{R-bookdown} in this sample book, which
was built on top of R Markdown and \textbf{knitr} \citep{xie2015}.

\chapter{Literature}\label{literature}

Here is a review of existing methods.

\chapter{Methods}\label{methods}

We describe our methods in this chapter.

\chapter{Applications}\label{applications}

Some \emph{significant} applications are demonstrated in this chapter.

\section{Example one}\label{example-one}

\section{Example two}\label{example-two}

\chapter{Final Words}\label{final-words}

We have finished a nice book.

\bibliography{book.bib,packages.bib}


\end{document}
